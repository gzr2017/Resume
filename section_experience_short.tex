% Awesome CV LaTeX Template
%
% This template has been downloaded from:
% https://github.com/huajh/huajh-awesome-latex-cv
%
% Author:
% Junhao Hua


%Section: Work Experience at the top
\sectionTitle{实习/项目经历}{\faCode}

\begin{experiences}

  \experience
  {2019年6月}   {U-Net 遥感图像语义分割}{中国地质大学(武汉)}{本科毕业设计}
  {2019年4月} {
    \begin{itemize}
      \item 使用 U-Net 网络识别遥感图像中的建筑物。
      \item 所提出的类别平衡交叉熵损失函数较交叉熵在 F1 Score 指标上提高 8.5\%,在召回率上较交叉熵提高 16\%。
      \item 在大幅遥感图像数据集 Inria Aerial Image Labeling Dataset 上建筑识别比例较交叉熵高,建筑识别更加完整。
      \item \faGithub:
            \link{https://github.com/gzr2017/UNet-AerialImageSegmentation}{gzr2017/UNet-AerialImageSegmentation}
    \end{itemize}
  }
  {语义分割, TensorFlow}
  \emptySeparator

  \experience
  {至今} {图像处理 100 问}{東北大学}{独立翻译}
  {2019年1月}    {
    \begin{itemize}
      \item  翻译了「画像処理100本ノック」的中文版「图像处理100问」。
      \item 该项目在 GitHub 上取得了 700+ stars。
      \item \faGithub:
            \link{https://github.com/gzr2017/ImageProcessing100Wen}{gzr2017/ImageProcessing100Wen}, \faGithub: \link{https://github.com/yoyoyo-yo/Gasyori100knock}{yoyoyo-yo/Gasyori100knock}
    \end{itemize}
  }
  {图像处理, 日语}
  \emptySeparator

  \experience
  { 2018年11月} {个人作品展示网站——RuoJunBai}{東北大学}{独立开发}
  {2018年12月}    {
    \begin{itemize}
      \item 为我的朋友白若珺制作个人绘画作品展示网站;
      \item 使用 JavaScript 完成简单网页人机交互;
      \item 虽然跟我关系不大,但是在申请季她被纽约视觉艺术学院录取了。
      \item \faLink: \link{https://ruojunbai.com} {Ruojun Bai}, \faGithub: \link{https://github.com/gzr2017/gzr2017.github.io} {gzr2017/gzr2017.github.io}, 
    \end{itemize}
  }
  {HTML, CSS, JavaScript}
  \emptySeparator

  \experience
  { 2019年3月} {Indoor Navigation}{東北大学}{交换留学}
  {2018年10月}    {
    \begin{itemize}
      \item 使用 HED-Net 分割并数字化纸质室内导航图。
      \item 入门深度学习。
    \end{itemize}
  }
  {图像分割, TensorFlow}
  \emptySeparator



  \experience
  {2018年6月} {烽火通信科技股份有限公司}{宽带业务产出线}{企业实习}
  {2018年9月 }    {
    担任系统工程师助理。
  }
  {企业实习}
  \emptySeparator

  \experience
  {至今} {个人博客——星山}{中国地质大学(武汉)}{独立开发}
  {2017年4月}    {
    \begin{itemize}
      \item  从2017年4月开始编写网页前端页面。
      \item 于2017年8月开始使用 Django 搭建后端服务器。
      \item 个人博客网站于2018年1月使用 NGINX+Gunicorn 上线。
      \item \faLink: \link{https://starmountain.ink} {星山}, \faGithub: \link{https://github.com/gzr2017/starmountain} {gzr2017/starmountain}
    \end{itemize}
  }
  {Django, MySQL, NGINX, Ubuntu, HTML, CSS, JavaScript}
  \emptySeparator

  \experience
  {2016年12月} {GoBang——Qt5 人机对战五子棋}{中国地质大学(武汉)}{独立开发}
  {2016年11月}    {
    \begin{itemize}
      \item  本科期间的小项目,使用 Qt 5 开发的人机对战五子棋。
      \item AI 算法基于决策树和 Alpha- Beta 剪枝。
      \item  最多可算五步棋。虽然我五子棋下得不好且电脑先手,但是我下不赢它。
      \item \faGithub: \link{https://github.com/gzr2017/GoBang} {gzr2017/GoBang}
    \end{itemize}
  }
  {Qt 5, 决策树}
  \emptySeparator

  \experience
  {2016年10月} {\emph{赴美带薪实习}}{Fishers Island, New York}{社会实践}
  {2016年6月}    {
   在 Fishers Island 的一家咖啡馆做收银员。锻炼了我的英语听力和口语能力。
  }
  {社会实践, 英语}
  \emptySeparator

  \experience
  {2016年9月} {中国地质大学(武汉)}{校报记者团}{摄影记者}
  {2015年9月}    {
    我一直以来都十分喜欢摄影。从拿到我的第一台相机到现在已经十年了。
    所以在大学里我加入了校报记者团担任摄影记者。也正是如此,研究生期间我也选择计算机视觉作为我的研究方向。
    我十分开心能将我的爱好和学习结合起来。不管能不能录取,希望你来Instagram看看我的摄影作品。
  }
  {摄影}
\end{experiences}
