% Awesome CV LaTeX Template
%
% This template has been downloaded from:
% https://github.com/huajh/huajh-awesome-latex-cv
%
% Author:
% Junhao Hua


%Section: Work Experience at the top
\sectionTitle{项目经历}{\faCode}
\begin{experiences}
  \experience
  {2019 年 06 月}   {U-Net 遥感图像语义分割}{中国地质大学(武汉)}{本科毕业设计}
  {2019 年 04 月} {
    \begin{itemize}
      \item 使用 U-Net 网络识别遥感图像中的建筑物。
      \item 所提出的类别平衡交叉熵损失函数较交叉熵在 F1-Score 指标上提高 8.5\%,在召回率上较交叉熵提高 16\%。
      \item 在大幅遥感图像数据集 Inria Aerial Image Labeling Dataset 上建筑识别比例较交叉熵高,建筑识别更加完整。
      \item \faGithub:
            \link{https://github.com/gzr2017/UNet-AerialImageSegmentation}{gzr2017/UNet-AerialImageSegmentation}
    \end{itemize}
  }
  {语义分割, TensorFlow}
  \emptySeparator

  \experience
  {至今} {图像处理 100 问}{東北大学}{独立翻译}
  {2019 年 01 月}    {
    \begin{itemize}
      \item  翻译了「画像処理100本ノック」的中文版「图像处理100问」。
      \item 该项目在 GitHub 上取得了 700+ stars。
      \item \faGithub:
            \link{https://github.com/gzr2017/ImageProcessing100Wen}{gzr2017/ImageProcessing100Wen}, \faGithub: \link{https://github.com/yoyoyo-yo/Gasyori100knock}{yoyoyo-yo/Gasyori100knock}
    \end{itemize}
  }
  {图像处理, 日语}
  \emptySeparator

  \experience
  {2018 年 12 月} {个人作品展示网站——RuoJunBai}{東北大学}{独立开发}
  {2018 年 11 月}    {
    \begin{itemize}
      \item 为我的朋友白若珺制作个人绘画作品展示网站;
      \item 使用 HTML5 完成简单网页人机交互。
      \item \faLink: \link{https://ruojunbai.com} {Ruojun Bai}, \faGithub: \link{https://github.com/gzr2017/gzr2017.github.io} {gzr2017/gzr2017.github.io},
    \end{itemize}
  }
  {HTML, CSS, JavaScript}
  \emptySeparator

  \experience
  {至今} {个人博客——星山}{中国地质大学(武汉)}{独立开发}
  {2017 年 04 月}    {
    \begin{itemize}
      \item  从 2017 年 4 月开始编写网页前端页面。
      \item 于 2017 年 8 月开始使用 Django 搭建后端服务器。
      \item 个人博客网站于 2018 年 1 月使用 NGINX+Gunicorn 上线。
      \item \faLink: \link{https://starmountain.ink} {星山}, \faGithub: \link{https://github.com/gzr2017/starmountain} {gzr2017/starmountain}
    \end{itemize}
  }
  {Django, MySQL, NGINX, Ubuntu, HTML, CSS, JavaScript}
  \emptySeparator

  \experience
  {2016 年 12 月} {GoBang——Qt5 人机对战五子棋}{中国地质大学(武汉)}{独立开发}
  {2016 年 11 月}    {
    \begin{itemize}
      \item  本科期间的小项目,使用 Qt 5 开发的人机对战五子棋。
      \item AI 算法基于决策树和 Alpha- Beta 剪枝。
      \item  最多可算五步棋。虽然我五子棋下得不好且电脑先手,但是我下不赢它。
      \item \faGithub: \link{https://github.com/gzr2017/GoBang} {gzr2017/GoBang}
    \end{itemize}
  }
  {Qt 5, 决策树}
  \emptySeparator
\end{experiences}
